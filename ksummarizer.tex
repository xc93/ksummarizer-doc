\documentclass{article}

\usepackage[utf8]{inputenc}
\usepackage{amsmath,amssymb,amsthm}
\usepackage{hyperref}
  \usepackage[capitalize]{cleveref}
\usepackage{comment}
\usepackage{ebproof}
\usepackage{framed}
\usepackage{lipsum}
\usepackage{mathtools}
\usepackage{xcolor}
\usepackage{xspace}



% styles

\newcommand{\df}[1]{\emph{#1}}
\newcommand{\wbar}{\overline}

\theoremstyle{plain}
  \newtheorem{theorem}{Theorem}
  \newtheorem{proposition}[theorem]{Proposition}  \newtheorem{lemma}[theorem]{Lemma}
  \newtheorem{corollary}[theorem]{Corollary}
\theoremstyle{definition}
  \newtheorem{definition}[theorem]{Definition}
%  \newtheorem{example}[theorem]{Example}
%\theoremstyle{remark}
%  \newtheorem{remark}[theorem]{Remark}
%  \newtheorem{notation}[theorem]{Notation}

% roman numbers
\makeatletter
  \newcommand{\Rnum}[1]{\expandafter\@slowromancap\romannumeral #1@}
\makeatother



% basic

\newcommand{\id}{\mathsf{id}}
\newcommand{\compose}{\circ}
\newcommand{\seqcompose}{\mathbin{\raise 0.6ex\hbox{\oalign{\hfil$\scriptscriptstyle      \mathrm{o}$\hfil\cr\hfil$\scriptscriptstyle\mathrm{9}$\hfil}}}}
\newcommand{\K}{\ensuremath{\mathbb{K}}\xspace}
\newcommand{\card}[1]{\mathsf{card}(#1)}
\newcommand{\codom}[1]{\mathsf{codom}(#1)}
\newcommand{\dom}[1]{\mathsf{dom}(#1)}
\newcommand{\fin}{\mathrm{fin}}
\newcommand{\pset}[1]{\mathcal{P}(#1)}
\newcommand{\prule}[1]{(\textsc{#1})}
\newcommand{\pr}[1]{\langle#1\rangle}
\newcommand{\Slash}{//\xspace}
\newcommand{\dnt}[1]{\llbracket #1 \rrbracket}
  \newcommand{\dntt}[2]{\dnt{#1}_{#2}}
\newcommand{\kto}{\curvearrowright}
\newcommand{\pto}{\rightharpoonup}
\newcommand{\ptof}{\pto_\fin}
\newcommand{\xtofrom}{\xlongleftrightarrow}
\newcommand{\hole}{\square}
\newcommand{\abs}[1]{\left|#1\right|}
\newcommand{\ld}{\,.\,}
\newcommand{\s}{\,}
\newcommand{\imp}{\rightarrow}
\newcommand{\dimp}{\leftrightarrow}
\newcommand{\simp}{\mathbin{-\!*}}
\newcommand{\cimp}{\mathbin{-\!\circ}}
\newcommand{\ev}[2]{|#1|_{#2}}
\newcommand{\FF}{\mathcal{F}}
\newcommand{\GG}{\mathcal{G}}
\newcommand{\lfp}[1]{\mathord{\mathbf{lfp}}(#1)}
\newcommand{\gfp}[1]{\mathord{\mathbf{gfp}}(#1)}
\newcommand{\restr}[2]{#1|_{#2}}
\newcommand{\ap}{\mathbin{\text{@}}}
  \newcommand{\apM}{\ap_M}
  \newcommand{\apMe}{\mathbin{\wbar{\apM}}}
  \newcommand{\apA}{\ap_A}
\newcommand{\FV}[1]{\mathit{FV(#1)}}
\newcommand{\cmark}{\ding{51}\xspace} % requires pifont
\newcommand{\xmark}{\ding{55}\xspace} % requires pifont
\newcommand{\cell}[2]{\left\langle #1 \right\rangle_{\text{#2}}}
 \newcommand{\cellm}[2]{\cell{...\ #1\ ...}{#2}}
 \newcommand{\cellf}[2]{\cell{#1\ ...}{#2}}
 \newcommand{\cells}[2]{\left\langle #1 \right\rangle^{*}_{\text{#2}}}
\newcommand{\inh}[1]{\top_{#1}}
\newcommand{\EV}{\mathit{EV}}
\newcommand{\SV}{\mathit{SV}}
\newcommand{\NN}{\mathbb{N}}
  \newcommand{\NNp}{\NN^+}
\newcommand{\ceil}[1]{\lceil#1\rceil}
\newcommand{\floor}[1]{\lfloor#1\rfloor}
\newcommand{\cln}{\,{:}\,}
\newcommand{\oto}{\to}
\newcommand{\To}{\Rightarrow}
\newcommand{\sand}{\mathbin{*}}
\newcommand{\proves}{\vdash}
\newcommand{\onext}{\mathord{\bullet}}
\newcommand{\anext}{\mathord{\circ}}
\newcommand{\ToExOne}{\To^{\exists}_{1}}
\newcommand{\ToAlOne}{\To^{\forall}_{1}}
\newcommand{\ToAl}{\To^{\forall}_{*}}
\newcommand{\sanext}{\anext_s}
\newcommand{\itSTOP}{\mathsf{STOP}}
\newcommand{\itNONSTOP}{\mathsf{NONSTOP}}
\newcommand{\varphitau}{\varphi^\tau}
\newcommand{\matchQ}{\triangleleft_?}
\newcommand{\unifyQ}{=_?}
\newcommand{\Er}{E_r}
\newcommand{\Ea}{E_a}
\newcommand{\Es}{E_s}
\newcommand{\tor}{\rightsquigarrow_r}
\newcommand{\toa}{\rightsquigarrow_a}
\newcommand{\tos}{\rightsquigarrow_s}

% derived
\newcommand{\varphiinit}{\varphi_\mathit{init}}
\newcommand{\varphifinal}{\varphi_\mathit{final}}
\newcommand{\Toexec}{\To_\mathsf{exec}}
\newcommand{\Toreach}{\To_\mathsf{reach}}
\newcommand{\varphipre}{\varphi_\mathit{pre}}
\newcommand{\varphipost}{\varphi_\mathit{post}}


% code

\newcommand{\code}[1]{\texttt{#1}\xspace}
\newcommand{\cc}[1]{{#1}}
\newcommand{\ccite}[3]{\mathbf{if}\ #1\ \mathbf{then}\ #2\ \mathbf{else}\ #3}
\newcommand{\ccwhile}[2]{\mathbf{while}\ #1\ \mathbf{do}\ #2}
\newcommand{\ccreturn}[1]{\mathbf{return}\ #1}
\newcommand{\ccif}{\mathbf{if}}
\newcommand{\ndnd}{\ \texttt{\&\&}\ }
\newcommand{\Cif}{C_{\ccif}}
\newcommand{\Cpct}{C_{\%}}
\newcommand{\Ceq}{C_{\textnormal{\texttt{==}}}}
\newcommand{\cceq}{{\ \textnormal{\texttt{==}}\ }}
\newcommand{\requires}{\text{requires}}
\newcommand{\KResult}{\mathsf{KResult}}

% non-logical symbols

\newcommand{\itplug}{\mathsf{plug}}
\newcommand{\ctxid}{\mathsf{id}}

% theories
\newcommand{\ThContexts}{\Gamma^{\mathsf{Context}}}

\title{The \K Summarizer}
\author{Runtime Verification, Inc.\\(written by Xiaohong Chen)}

\begin{document}

\maketitle

The purpose of this document is to give a formal specification of
the \K summarizer (\url{https://github.com/runtimeverification/erc20-verification/tree/master/ksummarize}).
We will investigate how the \K summarizer can be used to do semantics-based compilation (SBC) and formal verification. 

\section{Preliminaries}

Throughout this document, we assume that there is an 
underlying matching logic theory $\Gamma^L$ that defines the formal semantics
of a given programming language $L$. 
The provability symbol $\vdash$ in this document should always be understood as
\[\Gamma^L \vdash \varphi\]
i.e., $\varphi$ is provable with the formal semantics of $L$. 

The following standard notations will be used:
\begin{itemize}\renewcommand\labelitemi{--}
\item $\varphi$, $\psi$: an arbitrary pattern.
\item $t$: a term pattern, built from element variables and functional symbols.
\item $p$: a predicate pattern.
\item $t \land p$: a constraint term. 
\item $\tau \equiv [\varphi_1 / x_1 , \dots , \varphi_n / x_n]$: a substitution.
\item $\varphi\tau$ or $\varphi[\varphi_1 / x_1 , \dots , \varphi_n / x_n]$:
       applying the substitution to $\varphi$.
\item $\ceil{\varphi}$: the definedness pattern of $\varphi$.
\item $\onext \varphi$: ``one-path next'' of $\varphi$.
\item $\anext \varphi$: ``all-path next'' of $\varphi$, defined as
 $\anext \varphi \equiv \lnot \onext \lnot \varphi$.
\item $\itSTOP$: stopped/terminal states; abbreviation for $\anext \bot$.
\item $\itNONSTOP$: non-stopped states; abbreviation for $\onext \top$. 
\item $\sanext \varphi$: ``strongly all-path next'' of $\varphi$,
defined as $\sanext \varphi \equiv \anext \varphi \land \itNONSTOP$. 
\item $\varphi \ToExOne \varphi'$: ``one-step one-path execution''; abbreviation for $\varphi \imp \onext \varphi'$.
\item $\varphi \ToAlOne \varphi'$: ``one-step all-path execution'';
abbreviation for $\varphi \imp \sanext \varphi'$.
\item $\varphi \ToAl \varphi'$: ``all-path execution'';
abbreviation for $\varphi \imp \mu X \ld \varphi \lor \anext X$.
\item more to be added \ldots
\end{itemize}

\section{Matching and Unification}

In this section we formalize matching and unification as proving matching logic theorems.
Some definitions and results are from Arusoaie and Lucanu's paper \url{https://arxiv.org/pdf/1811.02835.pdf}. 

\subsection{Substitution Patterns}

\begin{definition}
Given a substitution
\[\tau \equiv [\varphi_1 / x_1 , \dots , \varphi_n / x_n] \]
we define a corresponding \emph{substitution pattern}
\[\varphitau \equiv (x_1 = \varphi_1) \land \dots \land (x_n = \varphi_n) \]
\end{definition}

\begin{proposition}
$\vdash \varphitau \imp (\psi = \psi\tau)$.
\end{proposition}
\begin{proof}
This (derived) proof rule is called \prule{Equality Elimination}. 
\end{proof}

\subsection{Matching}

\begin{definition}
\label{def:matching}
Let $\varphi$ and $\psi$ be two patterns. 
We say that $\varphi$ \emph{matches} $\psi$
if
\[\vdash \varphi \imp \exists \FV{\psi} \ld \psi \]
We say that $\{\sigma_1,\dots,\sigma_n\}$ is a \emph{complete solution}
to the matching problem
\[\varphi_1 \matchQ \psi_1, \dots, \varphi_m \matchQ \psi_m\] if
\[\vdash \left(\bigwedge_{i=1}^m \varphi_i \subseteq \psi_i \right)
  \dimp \varphi^{\tau_1} \lor \dots \lor \varphi^{\tau_n}
\]
\end{definition}

\begin{proposition}
For terms $t$ and $s$,
\Cref{def:matching} coincides with the classical definition of term matching.
\end{proposition}

\begin{proposition}
$\varphi$ matches $\psi$ if and only if
\[\vdash \left(\exists \FV{\varphi} \ld \varphi\right)
  \subseteq \left(\exists \FV{\psi} \ld \psi\right)
\]
\end{proposition}
\begin{proof}
By \Cref{def:matching}.
\end{proof}

\subsection{Unification}

\begin{definition}
\label{def:unification}
Let $\varphi$ and $\psi$ be two patterns. We say that $\varphi$ \emph{unifies}
with $\psi$ if
\[\vdash \ceil{\left(\exists \FV{\varphi} \ld \varphi\right)
  \land \left(\exists \FV{\psi} \ld \psi\right)}\]
We say that $\{\sigma_1,\dots,\sigma_n\}$ is a \emph{complete solution}
to the unification problem
\[\varphi_1 \unifyQ \psi_1, \dots, \varphi_m \unifyQ \psi_m\]
if
\[\vdash \left(\bigwedge_{i=1}^m \ceil{\varphi_i \land \psi_i}\right)
  \dimp \varphi^{\tau_1} \lor \dots \lor \varphi^{\tau_n}
\]
\end{definition}

\begin{proposition}
For terms $t$ and $s$,
\Cref{def:unification} coincides with the classical definition of term unification.
\end{proposition}

\subsection{Modulo Theories}

\Cref{def:matching,def:unification} work with underlying theories, 
in which case we obtain matching/unification modulo theories.

\section{\K Summaries}

\begin{definition}
A \K control-flow graph (abbreviated KCFG) $G = (V, \Er, \Ea, \Es)$ is a directed graph
with three types of edges where
\begin{itemize}
\item the vertex set $V$ is a set of patterns;
\item $\Er \subseteq V \times V$ is called the \emph{rewriting relation};
\item $\Ea \subseteq V \times V$ is called the \emph{abstracting relation};
\item $\Es \subseteq V \times V$ is called the \emph{splitting relation}. 
\end{itemize}
We write $\varphi \tor \psi$ 
($\varphi \toa \psi$ and $\varphi \tos \psi$, resp.)
for the three types of edges.
\end{definition}

\begin{definition}
A KCFG $G = (V, E, F)$ is \emph{sound} w.r.t. $\Gamma^L$ if
\begin{enumerate}
\item $\vdash \varphi \ToAl \psi$ for all $\varphi \tor \psi$;
\item $\vdash \varphi \imp \psi$ for all $\varphi \toa \psi$;
\item $\vdash \varphi \dimp \psi_1 \lor \dots \lor \psi_n$
 for all $\varphi_,\psi_1,\dots,\psi_n$ such that
 $\psi_1,\dots,\psi_n$ are all the $\Es$-successors of $\varphi$ in $G$.
\end{enumerate}
\end{definition}

Intuitively, $\varphi \ToAl \psi$ denotes a compilation of
many all-path execution steps. 

\end{document}