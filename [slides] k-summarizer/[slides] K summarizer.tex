\documentclass{beamer}

\usepackage[utf8]{inputenc}
\usepackage{hyperref}
  \usepackage[capitalize]{cleveref}
\usepackage{comment}
\usepackage{ebproof}
\usepackage{framed}
\usepackage{lipsum}
\usepackage{mathtools}
\usepackage{xcolor}
\usepackage{xspace}

% styles

\newcommand{\df}[1]{\emph{#1}}
\newcommand{\wbar}{\overline}

\theoremstyle{plain}
%  \newtheorem{theorem}{Theorem}
  \newtheorem{proposition}[theorem]{Proposition}
%\newtheorem{lemma}[theorem]{Lemma}
%  \newtheorem{corollary}[theorem]{Corollary}
%\theoremstyle{definition}
%  \newtheorem{definition}[theorem]{Definition}
%%  \newtheorem{example}[theorem]{Example}
%%\theoremstyle{remark}
%%  \newtheorem{remark}[theorem]{Remark}
%%  \newtheorem{notation}[theorem]{Notation}

% roman numbers
\makeatletter
  \newcommand{\Rnum}[1]{\expandafter\@slowromancap\romannumeral #1@}
\makeatother



% basic

\newcommand{\id}{\mathsf{id}}
\newcommand{\compose}{\circ}
\newcommand{\seqcompose}{\mathbin{\raise 0.6ex\hbox{\oalign{\hfil$\scriptscriptstyle      \mathrm{o}$\hfil\cr\hfil$\scriptscriptstyle\mathrm{9}$\hfil}}}}
\newcommand{\K}{\ensuremath{\mathbb{K}}\xspace}
\newcommand{\card}[1]{\mathsf{card}(#1)}
\newcommand{\codom}[1]{\mathsf{codom}(#1)}
\newcommand{\dom}[1]{\mathsf{dom}(#1)}
\newcommand{\fin}{\mathrm{fin}}
\newcommand{\pset}[1]{\mathcal{P}(#1)}
\newcommand{\prule}[1]{(\textsc{#1})}
\newcommand{\pr}[1]{\langle#1\rangle}
\newcommand{\Slash}{//\xspace}
\newcommand{\dnt}[1]{\llbracket #1 \rrbracket}
  \newcommand{\dntt}[2]{\dnt{#1}_{#2}}
\newcommand{\kto}{\curvearrowright}
\newcommand{\pto}{\rightharpoonup}
\newcommand{\ptof}{\pto_\fin}
\newcommand{\xtofrom}{\xlongleftrightarrow}
\newcommand{\hole}{\square}
\newcommand{\abs}[1]{\left|#1\right|}
\newcommand{\ld}{\,.\,}
\newcommand{\s}{\,}
\newcommand{\imp}{\rightarrow}
\newcommand{\dimp}{\leftrightarrow}
\newcommand{\simp}{\mathbin{-\!*}}
\newcommand{\cimp}{\mathbin{-\!\circ}}
\newcommand{\ev}[2]{|#1|_{#2}}
\newcommand{\FF}{\mathcal{F}}
\newcommand{\GG}{\mathcal{G}}
\newcommand{\lfp}[1]{\mathord{\mathbf{lfp}}(#1)}
\newcommand{\gfp}[1]{\mathord{\mathbf{gfp}}(#1)}
\newcommand{\restr}[2]{#1|_{#2}}
\newcommand{\ap}{\mathbin{\text{@}}}
  \newcommand{\apM}{\ap_M}
  \newcommand{\apMe}{\mathbin{\wbar{\apM}}}
  \newcommand{\apA}{\ap_A}
\newcommand{\FV}[1]{\mathit{FV(#1)}}
\newcommand{\cmark}{\ding{51}\xspace} % requires pifont
\newcommand{\xmark}{\ding{55}\xspace} % requires pifont
\newcommand{\cell}[2]{\left\langle #1 \right\rangle_{\text{#2}}}
 \newcommand{\cellm}[2]{\cell{...\ #1\ ...}{#2}}
 \newcommand{\cellf}[2]{\cell{#1\ ...}{#2}}
 \newcommand{\cells}[2]{\left\langle #1 \right\rangle^{*}_{\text{#2}}}
\newcommand{\inh}[1]{\top_{#1}}
\newcommand{\EV}{\mathit{EV}}
\newcommand{\SV}{\mathit{SV}}
\newcommand{\NN}{\mathbb{N}}
  \newcommand{\NNp}{\NN^+}
\newcommand{\ceil}[1]{\lceil#1\rceil}
\newcommand{\floor}[1]{\lfloor#1\rfloor}
\newcommand{\cln}{\,{:}\,}
\newcommand{\oto}{\to}
\newcommand{\To}{\Rightarrow}
\newcommand{\sand}{\mathbin{*}}
\newcommand{\proves}{\vdash}
\newcommand{\onext}{\mathord{\bullet}}
\newcommand{\anext}{\mathord{\circ}}
\newcommand{\ToExOne}{\To^{\exists}_{1}}
\newcommand{\ToAlOne}{\To^{\forall}_{1}}
\newcommand{\ToEx}{\To^{\exists}_{*}}
\newcommand{\ToAl}{\To^{\forall}_{*}}
\newcommand{\ToA}{\To^{\forall}}
\newcommand{\ToAs}{\To^{s,\forall}}
\newcommand{\ToE}{\To^{\exists}}
\newcommand{\eventually}{\diamond}
\newcommand{\sanext}{\anext_s}
\newcommand{\itSTOP}{\mathsf{STOP}}
\newcommand{\itNONSTOP}{\mathsf{NONSTOP}}
\newcommand{\varphitau}{\varphi^\tau}
\newcommand{\matchQ}{\triangleleft_?}
\newcommand{\unifyQ}{=_?}
\newcommand{\Er}{E_r}
\newcommand{\Ea}{E_a}
\newcommand{\Es}{E_s}
\newcommand{\tor}{\rightsquigarrow_r}
\newcommand{\toa}{\rightsquigarrow_a}
\newcommand{\tos}{\rightsquigarrow_s}
\newcommand{\Cfg}{\mathsf{Cfg}}
\newcommand{\ToCfg}{\To_\Cfg}

% derived
\newcommand{\varphiinit}{\varphi_\mathit{init}}
\newcommand{\varphifinal}{\varphi_\mathit{final}}
\newcommand{\Toexec}{\To_\mathsf{exec}}
\newcommand{\Toreach}{\To_\mathsf{reach}}
\newcommand{\varphipre}{\varphi_\mathit{pre}}
\newcommand{\varphipost}{\varphi_\mathit{post}}


% code

\newcommand{\code}[1]{\texttt{#1}\xspace}
\newcommand{\cc}[1]{{#1}}
\newcommand{\ccite}[3]{\mathbf{if}\ #1\ \mathbf{then}\ #2\ \mathbf{else}\ #3}
\newcommand{\ccwhile}[2]{\mathbf{while}\ #1\ \mathbf{do}\ #2}
\newcommand{\ccreturn}[1]{\mathbf{return}\ #1}
\newcommand{\ccif}{\mathbf{if}}
\newcommand{\ndnd}{\ \texttt{\&\&}\ }
\newcommand{\Cif}{C_{\ccif}}
\newcommand{\Cpct}{C_{\%}}
\newcommand{\Ceq}{C_{\textnormal{\texttt{==}}}}
\newcommand{\cceq}{{\ \textnormal{\texttt{==}}\ }}
\newcommand{\requires}{\text{requires}}
\newcommand{\KResult}{\mathsf{KResult}}

% non-logical symbols

\newcommand{\itplug}{\mathsf{plug}}
\newcommand{\ctxid}{\mathsf{id}}

% theories
\newcommand{\ThContexts}{\Gamma^{\mathsf{Context}}}

%\AtBeginSection[]
%{
%  \begin{frame}
%    \frametitle{Table of Contents}
%    \tableofcontents[currentsection]
%  \end{frame}
%}

\title{\K Summarizer: Foundations}
\author{Runtime Verification, Inc.}

\begin{document}

\frame{\titlepage}

\begin{frame}
\frametitle{Purpose}
Formalize the key concepts of the \K summarizer in matching logic.
\begin{itemize}
\item \K control-flow graphs
\item Basic blocks
\item Soundness and completeness
\end{itemize}
We work under the following assumptions:
\begin{itemize}
\item Formal semantics are deterministic
  (i.e., $\vdash \onext \varphi \imp \anext \varphi$)
\item All patterns are constrained terms: $t \land p$
\end{itemize}
\end{frame}

\begin{frame}
\frametitle{\K Control-Flow Graphs}
\begin{definition}
A \K control-flow graph (abbreviated KCFG) $G = (V, \Er, \Ea, \Es)$ is a finite directed graph
with three types of edges where
\begin{itemize}
\item the vertex set $V$ is a set of constrained terms;
\item $\Er \subseteq V \times V$ is called the \emph{rewriting relation};
\item $\Ea \subseteq V \times V$ is called the \emph{abstracting relation};
\item $\Es \subseteq V \times V$ is called the \emph{splitting relation}. 
\end{itemize}
We write $\varphi \tor \psi$ 
($\varphi \toa \psi$ and $\varphi \tos \psi$, resp.)
for the three types of edges.
\end{definition}
\end{frame}

\begin{frame}
\frametitle{Rewriting Edges}
$t_1 \land p_1 \tor t_2 \land p_2$ means
\begin{itemize}
\item finite- and at-least-one-step rewriting 
      \begin{equation}\label{eq:tor}
      \vdash t_1 \land p_1 \imp \onext \eventually (t_2 \land p_2)
      \end{equation}
\item $\eventually \varphi \equiv \mu X \ld \varphi \lor \onext X$
\item The next symbol $\onext$ enforces ``at-least-one-step''
\item Thanks to determinism, we can just use the ``one-path'' operators
      $\onext$ and $\eventually$.
\item \Cref{eq:tor} specifies a basic block.
\begin{itemize}
\item All the concrete instances of $t_1 \land p_1$ are covered:
\[\vdash \forall \bar{x}
   \ld \left( t_1 \land p_1 \imp \onext \eventually (t_2 \land p_2) \right)\]
\item Not getting stuck somewhere in the middle.
\item Determinism (by assumption)
\end{itemize}
\end{itemize}
\end{frame}

\begin{frame}
\frametitle{Abstracting Edges}
$t_1 \land p_1 \toa t_2 \land p_2$ means
\begin{itemize}
\item implication
      \begin{equation}\label{eq:toa}
      \vdash t_1 \land p_1 \imp \exists \bar{y} \ld t_2 \land p_2
      \end{equation}
      where $\bar{y} = \FV{rhs} \setminus \FV{lhs}$
\item The most common case is when
      $t_1 \equiv t_2[\bar{y}]_{\bar{p}}$ where $\bar{p}$ are the positions of $\bar{y}$ in $t_2$
\item It means that \Cref{eq:toa} has a witness substitution
      \[\pi = [t_{11} / y_1 \dots t_{1n} / y_n]\]
\item $t_1 \land p_1 \toa^\pi t_2 \land p_2$
\end{itemize}
\end{frame}

\begin{frame}
\frametitle{Splitting Edges}
$t \land p \tos t \land (p \land q_i)$ for $i = 1, 2, \dots, n$
\begin{itemize}
\item Complete Cases: $\vdash q_1 \lor \dots \lor q_n$
\item $t \land p \tos^{q_i} t \land (p \land q_i)$
\end{itemize}
\end{frame}

\begin{frame}
\frametitle{Review}
A KCFG $G = (V, \Er, \Ea, \Es)$ has
\begin{itemize}
\item V: a set of constrained terms (nodes)
\item $t_1 \land p_1 \tor t_2 \land p_2$: 
      basic block ($\ge 1$ steps)
\item $t \land p \tor^\pi t_2 \land p$ with a witness substitution $\pi$
\item $t \land p \tor^{q_i} t \land (p \land q_i)$ with a condition $q_i$
\begin{itemize}
\item $\vdash q_1 \lor \dots \lor q_n$
\end{itemize}
\end{itemize}
Termination Condition
\begin{itemize}
\item $\Phi_T$: a (user-provided) termination pattern\\
if not provided, $\Phi_T$ is $\anext \bot$ (e.g., when \code{<k>\,.\,</k>})
\item $t \land p$ has no successors iff $\vdash t \land p \imp \Phi_T$
\end{itemize}
\end{frame}

\begin{frame}
\frametitle{Semantics Derived from a KCFG}
Given a KCFG $G = (V, \Er, \Ea, \Es)$, we derive a new semantics
\begin{itemize}
\item for every $\varphi_1 \tos^{q} \varphi_2 \tor \varphi_3$, add
      \[ \text{rule} \quad \varphi_1 \To \varphi_3 \quad \text{requires} \  q\]
\item Let $\Gamma^G$ be the set of derived semantic rules.
\end{itemize}
\begin{theorem}
For any $t$ and its KCFG $G^t$, 
\[\Gamma^L \vdash t \To t' \quad\text{iff}\quad
  \Gamma^{G^t} \vdash t \To t'\]
\end{theorem}
\end{frame}

\end{document}