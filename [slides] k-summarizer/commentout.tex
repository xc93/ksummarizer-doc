\begin{frame}
\begin{definition}
A KCFG $G = (V, \Er, \Ea, \Es)$ is \emph{sound} if
\begin{enumerate}
\item (rewriting edges) $\vdash \varphi \ToAl \psi$ for all $\varphi \tor \psi$;
\item (abstracting edges) 
   $\vdash \varphi \imp \exists \bar{x} \ld \psi$ for all $\varphi \toa \psi$,
   where $\bar{x} = \FV{\psi} \setminus \FV{\varphi}$.
\item (splitting edges) $\vdash \varphi \dimp \psi_1 \lor \dots \lor \psi_n$
 for all $\varphi_,\psi_1,\dots,\psi_n$ such that
 $\psi_1,\dots,\psi_n$ are all the $\Es$-successors of $\varphi$ in $G$.
\end{enumerate}
\end{definition}
\end{frame}

\begin{frame}
\begin{example}

\end{example}
\end{frame}



\section{Rewriting in Matching Logic}

\begin{frame}
\frametitle{Matching Logic Patterns}
\[
\varphi \coloneqq x \mid X \mid \sigma(\varphi_1,\dots,\varphi_n)
\mid \bot \mid \varphi_1 \imp \varphi_2 \mid \exists x \ld \varphi
\mid \mu X \ld \varphi
\]
\end{frame}

\begin{frame}
\frametitle{Matching Logic Theories and Notations}
\begin{itemize}
\item $\ceil{\varphi}$, definedness/ceiling
\item $\floor{\varphi}$, totality/flooring
\item $\varphi_1 = \varphi_2$, equality
\item $\varphi_1 \subseteq \varphi_2$, set inclusion
\item $x \in \varphi$, membership
\item $\neg_s \varphi$, sorted negation
\item $\exists x \cln s \ld \varphi$ and
      $\forall x \cln s \ld \varphi$, sorted quantification
\item $\mu X \cln s \ld \varphi$ and $\nu X \cln s \ld \varphi$,
      sorted fixpoints
\end{itemize}
\end{frame}

\begin{frame}
\frametitle{Transition Systems}
Let $\Cfg$ be a distinguished sort of configurations.
Let us assume a configuration model
where $\ToCfg$ is a transition relation over configurations.
\begin{itemize}
\item $\onext \varphi$, one-path next\\
\begin{itemize}
\item $s \in \onext\varphi$ if there exists $s'$ such that
      $s \ToCfg s'$ and $s' \in \varphi$.
\item $\itNONSTOP \equiv \onext \top_\Cfg$,
      the set of non-terminating configurations
\item $\itSTOP \equiv \neg_\Cfg \itNONSTOP$,
      the set of terminating configurations
\end{itemize}
\item $\anext \varphi \equiv \neg_\Cfg \onext \neg_\Cfg \varphi$, all-path next
\begin{itemize}
\item $s \in \anext \varphi$ if for all $s'$ such that
      $s \To_\Cfg s'$, $s' \in \varphi$.
\item $\anext \bot$ is equivalent to $\itSTOP$.
\end{itemize}
\item Sometimes we want to enforce non-termination for all-path next.
\item $\sanext \varphi \equiv \anext \varphi \land \itNONSTOP$,
      strong all-path next.
\begin{itemize}
\item $\sanext \varphi$ is equivalent to $\anext \varphi \land \onext \varphi$.
\end{itemize}
\end{itemize}
\end{frame}

\begin{frame}
\frametitle{Rewrite Rules}
\begin{itemize}
\item $\varphi_1 \ToExOne \varphi_2 \equiv \varphi_1 \imp \onext \varphi_2$,
one-path one-step rewriting.
\begin{itemize}
\item The superscript $\exists$ means ``one-path''
\item The subscript $1$ means ``one-step''
\end{itemize}
\end{itemize}
A \emph{rewrite rule} is an axiom of the form
\[\varphi_1 \ToExOne \varphi_2\]
where $\varphi_1$ and $\varphi_2$ are patterns of sort $\Cfg$.

Semantically, $\varphi_1 \ToExOne \varphi_2$ holds
iff for all $s \in \varphi_1$ there exists $s' \in \varphi_2$ 
such that $s \To_\Cfg s'$.
\end{frame}

\begin{frame}
\frametitle{One-Path Rewriting}
\begin{itemize}
\item $\varphi_1 \ToE_2 \varphi_2 \equiv 
       \varphi_1 \imp \onext \onext \varphi_2$,
one-path two-step rewriting.
\item $\varphi_1 \ToE_3 \varphi_2 \equiv 
       \varphi_1 \imp \onext\onext \onext \varphi_2$,
one-path three-step rewriting.
\item \dots
\item $\eventually \varphi_2 \equiv \mu X \cln \Cfg \ld \varphi_2 \lor \onext X$
\begin{itemize}
\item $s \in \eventually\varphi_2$ if
      there exists a finite execution trace
      $s \To_\Cfg^n s'$ for $n \ge 0$ such that
      $s' \in \varphi_2$
\end{itemize}
\item $\varphi_1 \ToEx \varphi_2 \equiv
       \varphi_1 \imp \eventually\varphi_2$, one-path finite-step rewriting.
\begin{itemize}
\item It holds iff for all $s \in \varphi_1$
      there exists $s \To_\Cfg^n s'$ for $n \ge 0$ such that
      $s' \in \varphi_2$.
\end{itemize}
\end{itemize}
\end{frame}

\begin{frame}
\frametitle{All-Path Rewriting}
\begin{itemize}
\item $\varphi_1 \ToEx \varphi_2 \equiv
       \varphi_1 \imp \eventually\varphi_2$, one-path finite-step rewriting.
\begin{itemize}
\item It holds iff for all $s \in \varphi_1$
      there exists $s \To_\Cfg^n s'$ for $n \ge 0$ such that
      $s' \in \varphi_2$.
\end{itemize}
\item Question: what should ``all-path symbolic execution''
      $\varphi_1 \ToAl \varphi_2$ mean? 
\item (All-path reachability):
      for all $s \in \varphi_1$ and all complete 
      (i.e., finite and maximum) paths
      \[s = s_0 \To_\Cfg s_1 \To_\Cfg \dots \To_\Cfg s_n,\quad
       \text{$s_n$ terminating} \]
      there exists $0 \le m \le n$ such that $s_m \in \varphi_2$.
\begin{itemize}
\item Problem is that behaviors on infinite paths are entirely ignored. 
\end{itemize}
\item (More Reasonable?):
      for all $s \in \varphi_1$ and all complete or infinite
      paths
      \[s = s_0 \To_\Cfg s_1 \To_\Cfg \dots\]
      there exists $m \ge 0$ such that $s_m \in \varphi_2$.
\end{itemize}
\end{frame}

\begin{frame}
\frametitle{All-Path Rewriting}
\begin{itemize}
\item $\varphi_1 \ToAlOne \varphi_2 \equiv 
       \varphi_1 \imp \anext \varphi_2$,
all-path one-step rewriting.
\begin{itemize}
\item It holds iff for all $s \in \varphi_1$ and $s \To_\Cfg s'$,
      $s' \in \varphi_2$.
      Note that $s$ is allowed to be terminating.
\end{itemize}
\item $\varphi_1 \ToAs_1 \varphi_2 \equiv 
       \varphi_1 \imp \sanext \varphi_2$,
strong all-path one-step rewriting.
\begin{itemize}
\item In addition to $\varphi_1 \ToAlOne \varphi_2$, it requires
      all $s \in \varphi_1$ to be non-terminating.
\end{itemize}
\item $\varphi_1 \ToA_2 \varphi_2 \equiv 
       \varphi_1 \imp \anext\anext \varphi_2$,
one-path two-step rewriting.
\item $\varphi_1 \ToAs_2 \varphi_2 \equiv 
       \varphi_1 \imp \sanext\sanext \varphi_2$,
strong one-path two-step rewriting.
\item $\varphi_1 \ToAl \varphi_2 \equiv
       \varphi_1 \imp \mu X \ld \varphi_2 \lor \sanext X$, all-path
       rewriting.
\item $\varphi_1 \Toreach^\forall \varphi_2 \equiv
       \varphi_1 \imp \nu X \ld \varphi_2 \lor \sanext X$, all-path
       reachability.
\end{itemize}
\end{frame}

\begin{frame}
\frametitle{All-Path Rewriting}
\begin{lemma}
The following statements about $s$ are equivalent:
\begin{enumerate}
\item For all complete or infinite paths
      $s = s_0 \To_\Cfg s_1 \To_\Cfg \dots$
      there exists $m \ge 0$ such that $s_m \in \varphi$;
\item $s \in \ev{\mu X \ld \varphi_2 \lor \sanext X}{M}$
\end{enumerate}
\end{lemma}
\begin{proof}[Proof (TODO)]
Let $\xi$ and $\eta$ denote the set of all configurations
that satisfy (1) and (2), respectively.
Prove both $\xi \subseteq \eta$ and $\eta \subseteq \xi$.
Hint: use the Knaster-Tarski theorem to convert
$\eta$ into a big intersection.
\end{proof}
\end{frame}

\begin{frame}
\begin{lemma}
All-path rewriting implies all-path reachability:
$\vdash \varphi_1 \ToAl \varphi_2$ implies
$\vdash \varphi_1 \Toreach^\forall \varphi_2$.
\end{lemma}
\begin{proof}
Simply observe that all-path rewriting
\[\varphi_1 \ToAl \varphi_2 \equiv
       \varphi_1 \imp \mu X \ld \varphi_2 \lor \sanext X\]
is the least fixpoint while
all-path reachability 
\[\varphi_1 \Toreach^\forall \varphi_2 \equiv
       \varphi_1 \imp \nu X \ld \varphi_2 \lor \sanext X\]
is the greatest fixpoint. 
\end{proof}
\end{frame}

\section{\K Control-Flow Graphs}

\begin{frame}
\frametitle{\K Control-Flow Graphs}
\begin{definition}
A \K control-flow graph (abbreviated KCFG) $G = (V, \Er, \Ea, \Es)$ is a finite directed graph
with three types of edges where
\begin{itemize}
\item the vertex set $V$ is a set of patterns;
\item $\Er \subseteq V \times V$ is called the \emph{rewriting relation};
\item $\Ea \subseteq V \times V$ is called the \emph{abstracting relation};
\item $\Es \subseteq V \times V$ is called the \emph{splitting relation}. 
\end{itemize}
We write $\varphi \tor \psi$ 
($\varphi \toa \psi$ and $\varphi \tos \psi$, resp.)
for the three types of edges.
\end{definition}
\end{frame}

\begin{frame}
\begin{definition}
A KCFG $G = (V, \Er, \Ea, \Es)$ is \emph{sound} if
\begin{enumerate}
\item (rewriting edges) $\vdash \varphi \ToAl \psi$ for all $\varphi \tor \psi$;
\item (abstracting edges) 
   $\vdash \varphi \imp \exists \bar{x} \ld \psi$ for all $\varphi \toa \psi$,
   where $\bar{x} = \FV{\psi} \setminus \FV{\varphi}$.
\item (splitting edges) $\vdash \varphi \dimp \psi_1 \lor \dots \lor \psi_n$
 for all $\varphi_,\psi_1,\dots,\psi_n$ such that
 $\psi_1,\dots,\psi_n$ are all the $\Es$-successors of $\varphi$ in $G$.
\end{enumerate}
\end{definition}

In the current implementation of the \K summarizer, every 
pattern $\varphi$ 
has the form $t \land p$, called a constrained term,
where $t$ is a term
and $p$ is a predicate pattern.

We call such KCFGs \emph{regular}. 
\end{frame}

\begin{frame}
\begin{definition}
A regular KCFG is one that satisfies the following conditions:
\begin{enumerate}
\item It is sound.
\item All patterns are constrained terms. 
\item For all splitting edges
      $t_1 \land p_1 \tos t_2 \land p_2$ we have
      $t_1 \equiv t_2$ and $\vdash p_2 \imp p_1$.
\end{enumerate}
\end{definition}
A key property about a regular KCFG is that it is a complete summary
of all possible concrete executions.
\end{frame}

\begin{frame}

\begin{definition}
Given a KCFG, 
a flow $\pi$ is a sequence of the vertex patterns
following the three types of edges.
A flow instance $(\pi, \tau)$
is a flow $\pi$ associated with a substitution $\tau$
with $\dom{\tau} = \FV{\pi}$.
\end{definition}
\begin{theorem}[Tentative for now]
For any concrete execution of the original transition system,
there exists a corresponding flow instance
$(\pi,\tau)$ in the KCFG. 
\end{theorem}
\end{frame}

\section{Matching and Unification}

\begin{frame}

\frametitle{Substitution Patterns}

\begin{definition}
Given a substitution
\[\tau \equiv [\varphi_1 / x_1 , \dots , \varphi_n / x_n] \]
we define a corresponding \emph{substitution pattern}
\[\varphitau \equiv (x_1 = \varphi_1) \land \dots \land (x_n = \varphi_n) \]
\end{definition}

\begin{lemma}
$\vdash \varphitau \imp (\psi = \psi\tau)$.
\end{lemma}
\begin{proof}
This (derived) proof rule is called \prule{Equality Elimination}. 
\end{proof}
\end{frame}

\begin{frame}

\frametitle{Matching}

\begin{definition}
\label{def:matching}
Let $\varphi$ and $\psi$ be two patterns. 
We say that $\varphi$ \emph{matches} $\psi$
if
\[\vdash \varphi \imp \exists \FV{\psi} \ld \psi \]
We say that $\{\sigma_1,\dots,\sigma_n\}$ is a \emph{complete solution}
to the matching problem
\[\varphi_1 \matchQ \psi_1, \dots, \varphi_m \matchQ \psi_m\] if
\[\vdash \left(\bigwedge_{i=1}^m \varphi_i \subseteq \psi_i \right)
  \dimp \varphi^{\tau_1} \lor \dots \lor \varphi^{\tau_n}
\]
\end{definition}

\end{frame}

\begin{frame}

\frametitle{Matching}
\begin{lemma}
For terms $t$ and $s$,
the matching logic definition of matching
coincides with the classical term matching.
\end{lemma}

\begin{lemma}
$\varphi$ matches $\psi$ if and only if
\[\vdash \left(\exists \FV{\varphi} \ld \varphi\right)
  \subseteq \left(\exists \FV{\psi} \ld \psi\right)
\]
\end{lemma}

\end{frame}


\begin{frame}
\frametitle{Unification}

\begin{definition}
\label{def:unification}
Let $\varphi$ and $\psi$ be two patterns. We say that $\varphi$ \emph{unifies}
with $\psi$ if
\[\vdash \ceil{\left(\exists \FV{\varphi} \ld \varphi\right)
  \land \left(\exists \FV{\psi} \ld \psi\right)}\]
We say that $\{\sigma_1,\dots,\sigma_n\}$ is a \emph{complete solution}
to the unification problem
\[\varphi_1 \unifyQ \psi_1, \dots, \varphi_m \unifyQ \psi_m\]
if
\[\vdash \left(\bigwedge_{i=1}^m \ceil{\varphi_i \land \psi_i}\right)
  \dimp \varphi^{\tau_1} \lor \dots \lor \varphi^{\tau_n}
\]
\end{definition}
\end{frame}

\begin{frame}
\frametitle{Unification}

\begin{lemma}
For terms $t$ and $s$,
the matching logic definition of
unification coincides with classical term unification.
\end{lemma}

Both definitions of matching and unification
in matching logic
work with underlying theories, 
in which case we obtain the classical
matching/unification modulo theories.
\end{frame}


